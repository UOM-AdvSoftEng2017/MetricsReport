\section{Συμπεράσματα}

Για την αξιολόγηση της ποιότητας του συστήματος διαχείρισης κρατήσεων
ξενοδοχείου που αναπτύχθηκε, χρησιμοποιήθηκαν μετρικές μεγέθους,
επανάληψης κώδικα, πολυπλοκότητας, συνεκτικότητας και σύζευξης.

Αν και η εξέλιξη των περισσότερων μετρικών δείχνει αυξητικές τάσεις,
η αύξηση μπορεί να χαρακτηριστεί στις περισσότερες φορές ως φυσιολογική
που μπορεί να ερμηνευτεί από την αύξηση των γραμμών κώδικα του
συστήματος και τη συνεχή προσθήκη λειτουργικότητας σε αυτό.

Ωστόσο, υπάρχουν μετρικές που παρουσιάζουν ανησυχητικά υψηλές
τιμές. Η εξέτασή τους συγκλίνει στο ότι κάποια προσπάθεια αναδόμησης του
κώδικα θα μπορούσε να εφαρμοστεί για τη βελτίωση της ποιότητας του
λογισμικού. Η προσπάθεια αναδόμησης ουσιαστικά μπορεί να περιοριστεί
στις κλάσεις οντοτήτων DBManager και Reservation και σε μικρότερο βαθμό
στις κλάσεις γραφικής διασύνδεσης AddReservationFrame και
EditReservationFrame. Η αντιμετώπιση των προβλημάτων που παρουσιάζουν οι
κλάσεις αυτές θα οδηγήσει σε κατακόρυφη αύξηση της ποιότητας του
συστήματος, όπως αυτή εκφράζεται μέσα από τις μετρικές που
χρησιμοποιήθηκαν.

Επίσης, η δημιουργία ξεχωριστών κλάσεων unit test, αντί των πρόχειρων
ελέγχων που πραγματοποιούνταν στις μεθόδους main των περισσότερων
κλάσεων οντοτήτων, θα μείωνε ακόμα περισσότερο τις φαινομενικά υψηλές
τιμές κάποιων μετρικών για τις κλάσεις αυτές.

Συνολικά, οι κινήσεις που μπορούν να πραγματοποιηθούν για την αύξηση της
ποιότητας του συστήματος είναι στοχευμένες και εκτιμάται πως δεν θα
απαιτούσαν ιδιαίτερο κόπο από τα μέλη της ομάδας ανάπτυξης. Σίγουρα
αυτός ο κόπος θα ήταν σημαντικά λιγότερος από αυτόν που θα χρειαζόταν
σε αντίθετη περίπτωση αργότερα, με την περαιτέρω ανάπτυξη του συστήματος
και την αντιμετώπιση των προβλημάτων που θα προέκυπταν από τη μειωμένη
συντηρισημότητά του.

Αξίζει να αναφερθεί πως αν η αξιολόγηση της ποιότητας πραγματοποιούνταν
κατά τη διάρκεια της ανάπτυξης του λογισμικού, τουλάχιστον στο τέλος του
κάθε sprint, τα προβλήματα ποιότητας που φαίνεται να συσσωρεύονται με το
χρόνο, θα είχαν εντοπιστεί έγκαιρα. Έτσι θα μπορούσαν να εφαρμοστούν
μέθοδοι αναδόμησης του λογισμικού οι οποίες θα βελτίωναν την ποιότητά
του και οι οποίες δεν θα επέτρεπαν τη συνεχή αύξηση των τιμών των
μετρικών όπως τελικά παρατηρήθηκε. Το γεγονός αυτό υπογραμμίζει τη
σημαντικότητα της αξιολόγησης της ποιότητας του λογισμικού καθ' όλη τη
διάρκεια ανάπτυξής του και όχι μόνο στο τελικό προϊόν.
