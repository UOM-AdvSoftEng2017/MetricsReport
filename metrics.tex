\section{Μετρικές που χρησιμοποιήθηκαν}

Οι μετρικές που χρησιμοποιήθηκαν για την αξιολόγηση της ποιότητας του
συστήματος ήταν συνοπτικά οι:
\begin{itemize}
	\item Logical Lines Of Code (LLOC)
	\item Clone Coverage (CC)
	\item McCabe's Cyclomatic Complexity (McCC)
	\item Lach of Cohesion in Methods 5 (LCOM5)
	\item Coupling Between Object classes (CBO)
\end{itemize}

Όλες οι μετρικές υπολογίστηκαν τόσο σε επίπεδο έργου για κάθε commit που
πραγματοποιήθηκε, όσο και σε επίπεδο κλάσης/enumeration για το τέλος του
κάθε ενός από τα τρία sprint.

\subsection{Logical Lines Of Code (LLOC)}

Η μετρική LLOC μετρά τον συνολικό αριθµό των γραµµών του πηγαίου κώδικα
ενός τμήματος ή του προγράμματος, οπότε πρόκειται για \emph{μετρική
μεγέθους κώδικα}. Σχόλια και κενές γραμμές δεν υπολογίζονται. Οι
λογικές γραμμές κώδικα είναι ο αριθμός των γραμμών του εκτελέσιμου
κώδικα. Η μετρική LLOC είναι μια ένδειξη του βαθμού της
πολυπλοκότητας και συνεπώς της κατανοησιμότητας, της αναγνωσιμότητας
και της συντηρησιμότητας του κώδικα. Άυξηση των γραμμών κώδικα αποτελεί
ένδειξη αύξησης της πολυπλοκότητας του λογισμικού.

\subsection{Clone Coverage (CC)}

Πρόκειται για \emph{μετρική επανάληψης κώδικα}. Συγκεκριμένα πρόκειται
για μια αναλογία κώδικα που καλύπτεται από επαναλήψεις στον πηγαίο
κώδικα, με το μέγεθος του πηγαίου κώδικα, εκφρασμένη σε σχέση με τον
αριθμό των συντακτικών οντοτήτων (δηλώσεις, εκφράσεις, κ.λπ.).

\subsection{McCabe's Cyclomatic Complexity (McCC)}

Πρόκειται για \emph{μετρική πολυπλοκότητας}. Η πολυπλοκότητα της μεθόδου
που εκφράζεται ως ο αριθμός των ανεξάρτητων οδών ροής ελέγχου σε αυτό.
Αντιπροσωπεύει ένα κάτω φράγμα για τον αριθμό των πιθανών μονοπατιών
εκτέλεσης στον πηγαίο κώδικα και την ίδια στιγμή είναι ένα άνω όριο για
τον ελάχιστο αριθμό των περιπτώσεων δοκιμών που απαιτούνται για την
επίτευξη της πλήρους υποκαταστήματος κάλυψη δοκιμή.

Η τιμή της μετρικής υπολογίζεται ως ο αριθμός των παρακάτω εντολών συν
μία: if, for, foreach, while, do-while, case label. Επιπλέον, λογική «και»
(\&\&) και λογικό «ή» (||) εκφράσεις προσθέτουν επίσης 1 στην τιμή. Οι
παρακάτω εντολές δεν περιλαμβάνονται: else, switch, default label
(ανήκει στην εντολή switch), try, finally.


\subsection{Lack of Cohesion in Methods 5  (LCOM5)}

Πρόκειται για \emph{μετρική συνεκτικότητας}. Συγκεκριμένα πρόκειται για
μία εναλλακτική μετρική έλλειψης συνοχής. Αυτή η μετρική μέτρα την
έλλειψη συνοχής και υπολογίζει σε πόσες συνεκτικές κλάσεις θα μπορούσε
να σπάσει η κλάση. Η έλλειψη συνεκτικότητας σε μία κλάση υποδηλώνει ότι
πιθανόν να πρέπει να διασπαστεί σε περισσότερες κλάσεις. Χαμηλή
συνεκτικότητα συνεπάγεται πολυπλοκότητα, αυξημένη πιθανότητα εμφάνισης
σφαλμάτων, δυσκολία στη συντήρηση, δυσκολία στην επαναχρησιμοποίηση. Όσο
περισσότερες είναι οι συνεκτικές μέθοδοι, τόσο μεγαλύτερη είναι η
συνεκτικότητα και τόσο χαμηλότερη είναι η τιμή της μετρικής LCOM5.

\subsection{Coupling Between Object classes (CBO)}

Πρόκειται για \emph{μετρική σύζευξης}. Μετρά τον αριθμό των κλάσεων με
τις οποίες συνδέεται μία κλάση. Άρα δείχνει το πλήθος των κλάσεων από
τις οποίες εξαρτάται η τρέχουσα κλάση. Κάθε τέτοια εξάρτηση μπορεί να
είναι αμφίδρομη ή μονόδρομη οποιασδήποτε κατεύθυνσης. Επομένως υψηλό CBO
σημαίνει υψηλή σύζευξη (coupling) και αυτό δεν είναι επιθυμητό, καθώς
αποτρέπει την επαναχρησιμοποίηση κώδικα και ζημιώνει τον αρθρωτό
σχεδιασμό του προγράμματος. Όσο χαμηλότερη είναι η τιμή της μετρικής CBO
μίας κλάσης τόσο πιθανότερο είναι να μπορεί η τελευταία να
επαναχρησιμοποιηθεί. 

Κλάσεις που χρησιμοποιούν πολλές άλλες κλάσεις, εξαρτώνται σημαντικά από
το περιβάλλον τους, οπότε είναι δύσκολο να ελεγχθούν ή να
επαναχρησιμοποιηθούν και επιπλέον, είναι πολύ ευαίσθητες στις αλλαγές
που συμβαίνουν στο σύστημα. Η τιμή της CBO για μία κλάση είναι ίση με
τον αριθμό των άλλων κλάσεων στο σύστημα, με τις οποίες υπάρχει σύζευξη
(εξαιρώντας την κληρονομικότητα). Υπερβολική σύζευξη με άλλες κλάσεις
καθιστά την κλάση δυσκολότερα συντηρήσιμη και επαναχρησιμοποιήσιμη.
