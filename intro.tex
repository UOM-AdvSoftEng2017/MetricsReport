\section{Εισαγωγή}

Αντικείμενο της παρούσας εργασίας αποτελεί η ανάλυση της εξέλιξης του
λογισμικού διαχείρισης κρατήσεων ξενοδοχείου που αναπτύχθηκε στη
διάρκεια της 3ης εργασίας και η αξιολόγηση της ποιότητάς του.

Η αξιολόγηση της ποιότητας του συστήματος πραγματοποιήθηκε με τη χρήση
μετρικών λογισμικού. Οι μετρικές που χρησιμοποιήθηκαν ανήκουν στις
κατηγορίες μετρικών μεγέθους, επανάληψης κώδικα, πολυπλοκότητας,
συνεκτικότητας και σύζευξης, ώστε να διαπιστωθεί ολοκληρωμένα η ποιότητα
του λογισμικού.

Όλες οι μετρικές υπολογίστηκαν για κάθε commit που πραγματοποιήθηκε κατά τη
διάρκεια της ανάπτυξης του λογισμικού, χρησιμοποιώντας λογισμικό
στατικής ανάλυσης κώδικα. Σχεδιάστηκε η εξέλιξη των τιμών των μετρικών
συνολικά στο έργο σε σχέση με τον αύξοντα αριθμό commit, ανάλογα με το
χρόνο που αυτό πραγματοποιήθηκε. Επίσης, για το τέλος κάθε commit, όλες
οι μετρικές υπολογίστηκαν και παρουσιάζονται και σε επίπεδο κλάσης, ώστε
να διαπιστωθεί αν κάποιες κλάσεις παρουσιάζουν ιδιαίτερα προβλήματα
ποιότητας.
