\section{Εισαγωγή}

\subsection{Ποιότητα λογισμικού}

Σύμφωνα με το πρότυπο ISO 8402, «ποιότητα είναι το σύνολο των χαρακτηριστικών μιας οντότητας που της αποδίδουν την ικανότητα να ικανοποιεί εκφρασμένες και συνεπαγόμενες ανάγκες».

Τα διάφορα εργαλεία στατικής ανάλυσης κώδικα εφαρμόζουν διάφορες μετρήσεις με σκοπό να παράξουν έναν πιο ποιοτικό κώδικα. Οι μετρήσεις διεξάγονται με τη χρησιμοποίηση των μετρικών. Οι μετρικές χωρίζονται σε μετρικές προϊόντος (μετρικές ποιότητας κώδικα) που αφορούν τον προϊόν, δηλαδή τον πηγαίο κώδικα ή τις δηλώσεις ελέγχου, μετρικές έργου που σχετίζονται με την στρατηγική και την τακτική εκτέλεσης του έργου και καθορίζουν την ροή του έργου και τις τεχνικές που θα ακολουθηθούν και μετρικές διαδικασίας που αφορούν την διαδικασία κατασκευής του προϊόντος, όπως το σχεδιασμό, τη συγγραφή κώδικα και τους απαιτούμενους πόρους.

Η μέτρηση χαρακτηριστικών του λογισμικού είναι μια διαδικασία απαραίτητη για την εκτίμηση της κατάστασης του λογισμικού. Με τις μετρήσεις διασφαλίζεται πόσο καλά έχει συγγραφεί ο πηγαίος κώδικας, πόσο εφικτή είναι η επαναχρησιμοποίηση του σε άλλα έργα λογισμικού, καθώς και πόσο συντηρήσιμος είναι.

Οι πρώτες μετρικές προϊόντος παρουσιάστηκαν προς το τέλος της δεκαετίας του 70 από τον Halstead και τον McCabe. Τις ακολούθησε ένας μεγάλος αριθμός μετρικών που προτάθηκαν τα επόμενα χρόνια και συνεχίζουν να προτείνονται ακόμα και σήμερα. Στην πλειοψηφία τους, οι μετρικές προϊόντος περιορίζονται στη μέτρηση μεμονωμένων εσωτερικών χαρακτηριστικών του λογισμικού.

Η λειτουργικότητα ενός προγράμματος μπορεί να υλοποιηθεί είτε αυτό έχει καλή σχεδίαση, είτε κακή. Η ποιότητα σχεδίασης θα φανεί κατά τη διάρκεια της συντήρησης. Για μεγάλα πακέτα λογισµικού, τα προβλήµατα παρουσιάζονται ήδη από την φάση της σχεδίασης.  Ένα καλά σχεδιασµένο λογισµικό παρουσιάζει:
\begin{description}
\item[Aσθενή σύζευξη:] Ο όρος σύζευξη (coupling) περιγράφει το βαθµό που συνδέονται οι κλάσεις σε ένα πρόγραµµα. Επιδιώκουµε ασθενή σύζευξη, δηλαδή οι κλάσεις να είναι σχετικά ανεξάρτητες και να επικοινωνούν µέσω µικρών, απλών διεπαφών. Έτσι πετυχαίνουµε: να κατανοούµε µια κλάση, χωρίς να διαβάζουµε άλλες, να τροποποιούµε µια κλάση, χωρίς να αλλάζουµε άλλες.  Άρα αυξάνουµε τη συντηρησιµότητα.
\item[Ισχυρή συνοχή:] Ο όρος συνοχή (cohesion) αναφέρεται στο πλήθος και στην ποικιλία των καθηκόντων για το οποίο είναι υπεύθυνο ένα πρόγραµµα.  Η συνοχή αφορά σε κλάσεις και µεθόδους. Επιδιώκουµε ισχυρή συνοχή, δηλαδή κάθε τµήµα κώδικα να είναι υπεύθυνο για ένα καθήκον. Έτσι πετυχαίνουµε: α) να κατανοούµε εύκολα τι κάνει το κάθε τµήµα κώδικα, β) να αποφεύγουµε να γράφουµε (παρ)όµοιο κώδικα σε διαφορετικά σηµεία του προγράµµατος.  Άρα αυξάνουµε την επαναχρησιµοποίηση.
\end{description}

\subsection{Μετρικές κώδικα (code metrics)}

Οι μετρικές κώδικα χρησιμοποιούνται για την αξιολόγηση του κώδικα μιας εφαρμογής οι οποίες συνήθως περιλαμβάνουν μετρήσεις για τα παρακάτω:
\begin{itemize}
\item Αριθμός γραμμών κώδικα, που χρησιμοποιείται για την μέτρηση του μεγέθους του λογισμικού που αναπτύσσεται.
\item Κυκλωματική πολυπλοκότητα (Cyclomatic Complexity), που χρησιμοποιείται για την μέτρηση της πολυπλοκότητας ενός προγράμματος. Αυτό γίνεται υπολογίζοντας τον αριθμό των γραμμικά ανεξάρτητων μονοπατιών του πηγαίου κώδικα του προγράμματος.
\item Σφάλματα ανά γραμμή κώδικα.
\item Κάλυψη κώδικα, που χρησιμοποιείται για να υπολογιστεί το μέγεθος του κώδικα που έχει δοκιμαστεί.
Ταίριασμα κλάσεων, όπου μετρούνται οι εξαρτήσεις μεταξύ κλάσεων και αυτό γίνεται μέσω των παραμέτρων μιας κλάσης, των κλήσεων των μεθόδων της και των υλοποιήσεων interface.
\end{itemize}
